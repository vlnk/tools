\subsection*{la fonction \textit{TD\_get(int * T, int index, int * get)}}

On utilise la fonction \textit{TD\_get(int * T, int index, int * get)} pour récupérer une valeur contenue dans un sous-tableau selon un indice noté \textit{index} lié au tableau réel. On utilise alors la macro \textbf{BLOCK\_OWNER} pour récupérer le block qui est supposé contenir la valeur à cet indice. Cette macro permet d'obtenir le rang du block :
$$
block\_owner(index,p,n) = \frac{p \times (index + 1) - 1}{n}
$$

Lorsqu'on a récupéré le bon block, on a deux choix :
\begin{itemize}
    \item Soit c'est le block du processus qui appelle la fonction, dans ce cas on trouve la valeur directement.
    \item Soit le block est géré par un autre processus et dans ce cas on y accède avec la \textbf{RMA}.
\end{itemize}

\lstinputlisting[language=c, style=b&w, title={La fonction \textit{TD\_init(\dots)} abrégée}]{listings/get.c}

On accède à la \textbf{RMA} avec la fonction \verb+MPI_Get+ qui prend en paramètre la variable dans laquel la valeur trouvée est insérée, le rang du processus qui contient le bon block et l'indice correspondant du sous-tableau de ce même processus. Cet indice local est calculé en soustrayant le \textbf{FIRST} du bon block à l'indice cherché. 

\begin{figure}[h]
\centering
\caption*{Disposition des fenêtres locales pour \textit{MPI\_Get}}
\begin{tikzpicture}
%%%%%%%%%%%%%%%%%%%%%%%%%%%%%%%%%%%%%%%%%%%%%%%%%%%%%%%%
% STYLE
\tikzstyle{case}=[rectangle,draw,minimum width=1cm,minimum height=0.5cm,color=Black!40]
\tikzstyle{fcase}=[rectangle,draw,minimum width=1cm,minimum height=0.5cm]
\tikzstyle{txt}=[color=Black!40]

\tikzset{link/.style={->,>=stealth',color=Black!40}}

%%%%%%%%%%%%%%%%%%%%%%%%%%%%%%%%%%%%%%%%%%%%%%%%%%%%%%%%
% NODES
\node[txt] (t) at (-1, 0) {$T$};

\node[case] (T00) at (0, 0) {$4$};
\node[case] (T01) at (1, 0) {$7$};
\node[case] (T02) at (2, 0) {$5$};
\node[case] (T03) at (3, 0) {$9$};

\node[txt] (R0) at (5, 0) {$\}$ rank 0};

\node[case] (T10) at (0, -1.5) {$3$};
\node[case] (T11) at (1, -1.5) {$6$};
\node[case] (T12) at (2, -1.5) {$1$};
\node[case] (T13) at (3, -1.5) {$8$};

\node[txt] (R1) at (5, -1.5) {$\}$ rank 1};

\node[case] (T20) at (0, -3) {$8$};
\node[case] (T22) at (2, -3) {$5$};
\node[fcase] (T21) at (1, -3) {$2$};
\node[case] (T22) at (3, -3) {$0$};

\node[txt, color=Black] (R2) at (5, -3) {$\}$ block\_owner};
\node[txt, color=Black] (R3) at (1, -3.5) {indice $= 8$};
\node[txt, color=Black] (R4) at (1, -4) {indice local $= 1$};

%%%%%%%%%%%%%%%%%%%%%%%%%%%%%%%%%%%%%%%%%%%%%%%%%%%%%%%%
% LINKS
\draw[link] (t) -- (T00);

\draw[link] (T03.east) to [out=320,in=140] (T10.west);
\draw[link] (T13.east) to [out=320,in=140] (T20.west);

\end{tikzpicture}
\end{figure}

On récupère alors la valeur cherchée dans la variable \textit{get}. Si l'indice indiqué est invalide par rapport au tableau, la variable \textit{get} renvoie la valeur $-1$. Dans le programme, la fonction \textit{TD\_get()} est appelée par une processus arbitrairement choisi (ici, c'est $p-1$).

Par contre, l'utilisation de \verb+MPI_Get+ et de la \textbf{RMA} oblige les processus à se synchroniser avec \verb+MPI_Win_fence+ avant et après l'appel de fonction utilisant la \textbf{RMA}. Étant donné que cette fonction n'est pas attachée à un groupe \verb+MPI_Comm+, il est obligatoire que tous les processus franchissent ces barrières ce qui sera contraignant pour le tri en parallèle.

\begin{lstlisting}[language=c, style=b&w, title={Utilisation de \textit{MPI\_Win\_fence} pour l'utilisation de RMA}]
MPI_Win_fence((MPI_MODE_NOPUT | MPI_MODE_NOPRECEDE), win);
if (rank == p - 1) TD_get(T, indice_get, &x);
MPI_Win_fence(MPI_MODE_NOSUCCEED, win);
\end{lstlisting}

\subsection*{la fonction \textit{TD\_put(int * T, int index, int put)}}
La fonction \textit{TD\_get(int * T, int index, int put)} utilise le même principe que la fonction \textit{TD\_get()} sauf que la fonction \verb+MPI_Put+ est utilisée à la place de \verb+MPI_Get+ afin de remplacée la valeur à l'indice cherché.