\subsection*{la fonction \textit{TD\_init(int * T)}}

On alloue une partie du tableau que sera traitée par processus ; dans cette optique avec \verb+MPI_Type_size(MPI_INT, &size_of_int)+ on récupère la taille d'un \textit{integer} qui peut être spécifique à chaque processus puis on alloue le tableau T de taille \textit{size} grâce à la fonction \textit{MPI} \verb+MPI_Alloc_mem(size_of_int*size, MPI_INFO_NULL, &T)+. 

\lstinputlisting[language=c, style=b&w, title={La fonction \textit{TD\_init(int* T)} abrégée}]{listings/init.c}

\begin{figure}[h]
\centering
\caption*{Disposition des fenêtres locales en \textbf{RMA} pour 3 processus et un tableau de taille $12$}
\begin{tikzpicture}
%%%%%%%%%%%%%%%%%%%%%%%%%%%%%%%%%%%%%%%%%%%%%%%%%%%%%%%%
% STYLE
\tikzstyle{case}=[rectangle,draw,minimum width=1cm,minimum height=0.5cm]
\tikzstyle{op}=[circle,draw]
\tikzstyle{txt}=[]

\tikzset{link/.style={->,>=stealth'}}

%%%%%%%%%%%%%%%%%%%%%%%%%%%%%%%%%%%%%%%%%%%%%%%%%%%%%%%%
% NODES
\node[txt] (t) at (-1, 0) {$T$};

\node[case] (T00) at (0, 0) {$4$};
\node[case] (T01) at (1, 0) {$7$};
\node[case] (T02) at (2, 0) {$5$};
\node[case] (T03) at (3, 0) {$9$};

\node[txt] (R0) at (5, 0) {$\}$ rank 0};

\node[case] (T10) at (0, -1.5) {$3$};
\node[case] (T11) at (1, -1.5) {$6$};
\node[case] (T12) at (2, -1.5) {$1$};
\node[case] (T13) at (3, -1.5) {$8$};

\node[txt] (R1) at (5, -1.5) {$\}$ rank 1};

\node[case] (T20) at (0, -3) {$8$};
\node[case] (T21) at (1, -3) {$2$};
\node[case] (T22) at (2, -3) {$5$};
\node[case] (T22) at (3, -3) {$0$};

\node[txt] (R2) at (5, -3) {$\}$ rank 2};

%%%%%%%%%%%%%%%%%%%%%%%%%%%%%%%%%%%%%%%%%%%%%%%%%%%%%%%%
% LINKS
\draw[link] (t) -- (T00);

\draw[link] (T03.east) to [out=320,in=140] (T10.west);
\draw[link] (T13.east) to [out=320,in=140] (T20.west);

\end{tikzpicture}
\end{figure}

Dans la fonction \textit{TD\_init()}, chaque processeur crée un tableau de la taille du block dont il est responsable, et initialise aléatoirement les valeurs. Aucun tableau \emph{complet} n'est crée, on travaille avec des indices allant de $0$ à la taille du tableau $-1$, mais pour travailler directement avec les valeurs il faut accéder à la mémoire du processeur responsable, et donc adapter l'indice pour ce processeur.

La gestion de la \textbf{remote-memory access} permet aux processus de partager une partie de leur mémoire avec les autres processus. Cela facilite les différents accès mémoire entre les processus. On instancie la \textbf{RMA} avec la fonction \verb+MPI_Win_create+ qui permet la création d'une fenêtre accessible par les processus. Chaque fenêtre local prend la taille du tableau \textit{c'est-à-dire} la taille du bloc multiplié par la taille d'une \textit{integer} en octets. La taille du bloc est calculé avec les macros \textbf{FIRST} et \textbf{LAST} telles que :
\begin{align*}
first(r,p,n) &= r \times \frac{n}{p}\\
last(r,p,n) &= first(r+1,p,n) - 1\\
size(r,p,n) &= last(r,p,n) - first(r,p,n) + 1
\end{align*}