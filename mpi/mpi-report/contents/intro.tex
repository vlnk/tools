\section*{Exécution du programme}
Le programme s'exécute à partir de \verb+8inf856-14@dim-linuxmpi1+ en exécutant le script \verb+tp3/mpi_run+. Le script compile le programme \verb+tp3.c+ avec \verb+mpicc+, transfert l'exécutable sur les comptes \verb+dim-linuxmpi+ de $2$ à $6$, et exécute le programme sur les $6$ machines via \verb+mpirun+.

\begin{center}
\begin{lstlisting}[language=bash, style=b&w, title={Script pour lancer le programme \emph{tp3} avec mpi}]
NODES=6
if [ -z $1 ] || [ -z $2 ] || [ -z $3 ]; then
    echo "USE: $0 <app> <number_of_node> <args>"
elif [ $2 -gt $NODES ] && [ $2 -lt 0 ]; then
    echo "ERROR: number of nodes invalid"
else
    if mpicc -o $1 $1.c; then
        while [ $NODES -gt 1 ]; do
            scp -r "$1" "8inf856-14@dim-linuxmpi$NODES.uqac.ca:$1"
            let NODES=NODES-1
        done

        clear
        mpirun -np $2 $1 $3
    fi
fi
\end{lstlisting}
\end{center}

Ainsi, le commande à exécuter pour lancer le programme est :
\begin{center}
\verb+tp3/mpi_run.sh tp3/tp3 <nb processus> <taille du tableau>+ 
\end{center}
à partir du la racine \verb+[8inf856-14@dim-linuxmpi1 ~]+.