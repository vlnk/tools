\subsection*{algorithme}
Après avoir déroulé l'arbre récursif de l'algorithme \verb+ST1+, on a trouvé une problématique similaire à la question 1. C'est pourquoi, nous sommes arrivés à un algorithme proposant un mécanisme similaire.

\begin{lstlisting}[mathescape, style=cilk, title=Algorithme ittératif]
int * add($T$, $n$)
    int $i$,$j$,$l$;

    if $n = 2$ then
        $T[2] += T[1]$
    else
        for $i = 2$ to n by $i \times 2$ do
            parallel for $j = 1$ to $\frac{n}{i}$ do
                parallel for $l = 1$ to $i$ do
                    $T[j \times i + l] = T[j \times i + \frac{i}{2} - 1] + T[j \times i + l]$
\end{lstlisting}

On déroule alors l'algorithme selon le schéma ci-dessous, on voit alors qu'on peut paralléliser les traitement de chaque sous-tableau.

\begin{figure}[h]
\centering
\caption*{Schéma d'exécution de l'algorithme pour un tableau de taille $8$}
\begin{tikzpicture}
%%%%%%%%%%%%%%%%%%%%%%%%%%%%%%%%%%%%%%%%%%%%%%%%%%%%%%%%
% STYLE
\tikzstyle{case}=[rectangle,draw,minimum width=1cm,minimum height=0.5cm]
\tikzstyle{separator}=[rectangle,draw,minimum width=0.1cm,minimum height=0.5cm,fill=black]
\tikzstyle{op}=[circle,draw]
\tikzstyle{txt}=[]

\tikzset{link/.style={->,>=stealth'}}

%%%%%%%%%%%%%%%%%%%%%%%%%%%%%%%%%%%%%%%%%%%%%%%%%%%%%%%%
% NODES
\node[txt] (t) at (-1, 0) {t};

\node[case] (T00) at (0, 0) {$4$};
\node[case] (T01) at (1, 0) {$7$};
\node[case] (T02) at (2, 0) {$5$};
\node[case] (T03) at (3, 0) {$9$};
\node[case] (T04) at (4, 0) {$2$};
\node[case] (T05) at (5, 0) {$1$};
\node[case] (T06) at (6, 0) {$6$};
\node[case] (T07) at (7, 0) {$3$};

\node[separator] (S00) at (1.5, 0) {};
\node[separator] (S01) at (3.5, 0) {};
\node[separator] (S02) at (5.5, 0) {};
\node[op] (O0) at (8, -0.75) {$+$};

\node[case] (T10) at (0, -1.5) {$4$};
\node[case] (T11) at (1, -1.5) {$11$};
\node[case] (T12) at (2, -1.5) {$5$};
\node[case] (T13) at (3, -1.5) {$14$};
\node[case] (T14) at (4, -1.5) {$2$};
\node[case] (T15) at (5, -1.5) {$3$};
\node[case] (T16) at (6, -1.5) {$6$};
\node[case] (T17) at (7, -1.5) {$9$};

\node[separator] (S10) at (3.5, -1.5) {};
\node[op] (O1) at (8, -2.25) {$+$};

\node[case] (T20) at (0, -3) {$4$};
\node[case] (T21) at (1, -3) {$11$};
\node[case] (T22) at (2, -3) {$16$};
\node[case] (T23) at (3, -3) {$25$};
\node[case] (T24) at (4, -3) {$2$};
\node[case] (T25) at (5, -3) {$3$};
\node[case] (T26) at (6, -3) {$9$};
\node[case] (T27) at (7, -3) {$12$};

\node[op] (O2) at (8, -3.75) {$+$};

\node[case] (T30) at (0, -4.5) {$4$};
\node[case] (T31) at (1, -4.5) {$11$};
\node[case] (T32) at (2, -4.5) {$16$};
\node[case] (T33) at (3, -4.5) {$25$};
\node[case] (T34) at (4, -4.5) {$27$};
\node[case] (T35) at (5, -4.5) {$28$};
\node[case] (T36) at (6, -4.5) {$34$};
\node[case] (T37) at (7, -4.5) {$37$};

%%%%%%%%%%%%%%%%%%%%%%%%%%%%%%%%%%%%%%%%%%%%%%%%%%%%%%%%
% LINKS
\draw[link] (t) -- (T00);
\draw[link] (T00) -- (T11);
\draw[link] (T02) -- (T13);
\draw[link] (T04) -- (T15);
\draw[link] (T06) -- (T17);

\draw[link] (T11.270) -- (T22.90);
\draw[link] (T11.270) -- (T23.90);
\draw[link] (T15.270) -- (T26.90);
\draw[link] (T15.270) -- (T27.90);

\draw[link] (T23.270) -- (T34.90);
\draw[link] (T23.270) -- (T35.90);
\draw[link] (T23.270) -- (T36.90);
\draw[link] (T23.270) -- (T37.90);
\end{tikzpicture}
\end{figure}

Ainsi notre vision séquentielle est la division successive du tableau en sous tableaux de taille $2$, puis $4$, jusqu'à $n$. Ensuite, on sépare chaque morceau en 2 et tous les éléments de la première partie reçoivent additionnellement les éléments d'indice correspondant de la seconde partie.

Étant donné que la première boucle ne peut pas être parallélisée, la durée de cet algorithme est :
\begin{equation} \label{eq3}
\begin{split}
T_\infty(n) & = log \, n \times (log \, n + log \, n)\\ 
 & = \mathcal{O} (log^2 \, n)
\end{split}
\end{equation}

\subsection*{tests}

On effectue alors une batterie de tests représentant les temps d'exécution en fonction du nombre de valeurs afin de comparer les deux algorithmes.

\begin{center}
    \begin{tabular}{ | l | l | l | l |}
    \hline
    \multicolumn{2}{| c}{récursif (sp1)}   & \multicolumn{2}{| c |}{itératif (sp2)} \\ \hline
    nombre de valeurs   & temps réel (en sec)     & nombre de valeurs     & temps réel (en sec) \\ \hline 
	$8$ 			& $0.001009$	& $8$ 			& $0.000758$ \\
	$16$ 			& $0.000536$	& $16$ 			& $0.001141$ \\
	$256$ 			& $0.000716$	& $256$ 		& $0.000974$ \\
	$4048$ 			& $0.000752$	& $4048$ 		& $0.001975$ \\
	$16384$ 		& $0.001063$	& $16384$ 		& $0.003936$ \\
	$32768$ 		& $0.001524$	& $32768$ 		& $0.007158$ \\
	$65536$ 		& $0.001834$	& $65536$ 		& $0.013404$ \\
	$131072$ 		& $0.003007$	& $131072$ 		& $0.026134$ \\
	$262144$ 		& $0.006430$	& $262144$ 		& $0.052054$ \\
	$524288$ 		& $0.011124$	& $524288$ 		& $0.103591$ \\
	$1048576$ 		& $0.021289$	& $1048576$ 	& $0.206191$ \\
	$2097152$ 		& $0.036996$	& $2097152$ 	& $0.432396$ \\
	$4194304$ 		& $0.072508$	& $4194304$ 	& $0.828926$ \\
	$8388608$ 		& $0.145081$	& $8388608$ 	& $1.674089$ \\
	$16777216$ 		& $0.287818$	& $16777216$ 	& $3.378797$ \\
	$33554432$ 		& $0.576497$	& $33554432$ 	& $6.997660$ \\
	$67108864$ 		& $1.150001$	& $67108864$ 	& $13.552039$ \\
	$134217728$ 	& $2.301539$	& $134217728$ 	& $27.476417$ \\
	$268435456$ 	& $4.607856$	& $268435456$ 	& $54.842712$ \\
	$536870912$ 	& $39.832040$	& $536870912$ 	& $112.819337$ \\ \hline
    \end{tabular}
\end{center}

\subsection*{analyse}
Théoriquement, notre algorithme itératif devrait s'exécuter aussi vite que le récursif (à une constante près) étant donné que leurs durées valent toutes deux $T_\infty(n) = \mathcal{O} (log^2 \, n)$. D'après nos résultats, on constate qu'avec un $n$ assez grand on a la précence d'un facteur entre les temps d'exécution (ici, cette valeur est environ $13$). 

En cherchant ce qui ce se passe au niveau des threads, nous avons remarqué que dans la première boucle parallélisée il y avait $8$ threads en activité, mais dans la seconde il y en a toujours qu'un seul.

\begin{lstlisting}[style=cilk, title=Parallélisation des boucles itératives avec les commandes OpenMP]
for (i = 2; i <= n; i*=2)
{
	#pragma omp parallel for private(j)
	for (j = 0; j < n/i; ++j)
	{
		printf("number of threads %d\n", omp_get_num_threads()); /*display "8"*/
		printf("number of the thread : %d\n", omp_get_thread_num());
        
		#pragma omp parallel for private(l)
		for (l = i/2; l < i; ++l);
        {
			T[j*i + l] = T[j*i + i/2 - 1] + T[j*i + l];
            
            printf("number of threads %d\n", omp_get_num_threads()); /*display "1"*/
            printf("number of the thread : %d\n", omp_get_thread_num());
		}
	}
}
\end{lstlisting}


N'ayant pas une infinité de threads à disposition, la première parallélisation les occupe tous, et la deuxième n'a donc à disposition que le seul thread associé à l'itération de la boucle précédente ; on revient alors à du séquentiel pour cette boucle tandis que la première boucle est en séquentiel par \emph{blocs}. La durée revient alors, dans le cas où la première boucle est parallélisée avec sufisament de threads à $T_\infty(n) = \mathcal{O} (n \times log \, n)$ (ce que est beaucoup plus grand que du $\mathcal{O}(log^2 \, n)$).

De la même manière, dans la version récursive, on dispose également que de $8$ threads. Dans celle-ci tous les threads se partagent chaque boucle mais n'étant que $8$ avec des boucles allant de $0$ à $\frac{n}{2}$, on obtient $\frac{n}{16}$ blocs à exécuter en séquentiel (les threads sont en parallèle dans les blocs). Ce qui implique que ces boucles ne sont plus en $\mathcal{O} (log \, n)$ mais en $\mathcal{O} (n)$. Ainsi en pratique on a une durée de $T_\infty(n) = \mathcal{O} (n \times log \, n)$.